\documentclass{amsart}
\usepackage{amsmath}
\usepackage{amssymb}
\usepackage{latexsym}
%\usepackage{booktabs}
\usepackage[margin=1in]{geometry}

\usepackage[draft,layout=footnote]{fixme}
\FXRegisterAuthor{ss}{ass}{SS}

\newcommand{\CC}{\mathbb{C}}

\title[\small{Scattering Diagrams and Greedy Bases for rank-2 Cluster Algebras}]
  {AIM SQuaRE Proposal\\ 
  \small{Scattering Diagrams and Greedy Bases for rank-2 Cluster Algebras}}

\begin{document}
  \maketitle
  
  \section*{Overview}
    The goal of this AIM SQuaRE Proposal is to complete a project started at the
    AMS's Mathematics Research Communities program in June 2014. 
    
    Specifically, during the conference on cluster algebras, a study group
    addressed the problem of understanding the relation among two seemingly
    different constructions of bases in rank 2: the \emph{greedy basis} of
    Lee-Li-Zelevinsky and the \emph{theta basis} of
    Gross-Hacking-Keel-Kontsevich. The group managed to show that the two
    constructions are equivalent but our bijection is not explicit. The aim for
    this SQuaRE meeting is to finalize the details in our proof and construct an
    explicit bijection between the objects involved in each basis.
  
  \subsection*{Participants}
    The participants to the meeting will be 
    \sswarning{We need to decide on who is coming and move the rest to the next
      list. The cap is to 6 participants.}
    \begin{itemize}
      \item Man Wai Cheung (Ph.D. Candidate, UCSD Department of Mathematics;
          Visiting student, University of Cambridge)
      \item Maria Angelica Cueto (NSF Postdoctoral Fellow, Columbia University)
      \item Mark Gross (Professor, University of Cambridge)
      \item Gregory Muller (Assistant Professor, University of Michigan)
      \item Gregg Musiker (Assistant Professor, University of Minnesota)
      \item Dylan Rupel (Postdoctoral Teaching Associate, Northeastern
          University)
      \item Salvatore Stella (Postdoc, North Carolina State University)
      \item Harold Williams (RTG Postdoctoral Instructor, University of 
          Texas at Austin) 
    \end{itemize}
    In addition to the above the original group included also
    \begin{itemize}
      \item
    \end{itemize}
    Even if they will not be present at this SQuaRE meeting they remain integral
    member of this study group. 
    \sswarning{Shall we mention that we might call those that will not attend to
        work with the rest of the participants remotely?}
  
  \subsection*{Timeline}
    \sswarning{The online form does not say anywere that we need to give a time
      for this SQuaRE meeting but I guess it is in our best interest to do so.}

  \section*{Project description}

    A central issue in the theory of cluster algebras is to define a natural
    ``canonical'' basis of a cluster algebra.  Indeed, the original inspiration
    for the discovery of cluster algebras was the recognition of certain
    combinatorial structures underlying Lusztig's dual canonical basis in the
    coordinate ring of a unipotent group.  Thus one of the goals of the subject
    is to understand this particular basis on a more abstract level and
    generalize it to other interesting cluster algebras.  There are by now many
    different approaches to this problem, leading in fact to a number of
    different bases in any given cluster algebra.

    Our goal and main accomplishment during the MRC program  was to show the
    equivalence of two seemingly different constructions of bases in rank 2
    cluster algebras.  One of these, the \emph{greedy basis} of Lee-Li-Zelevinsky, is
    described in terms of certain combinatorial objects called compatible pairs.
    Each greedy basis element is associated with a Dyck path in some rectangular
    grid, and a compatible pair is a configuration of subpaths satisfying
    certain conditions.  The expansion of the basis element in an initial
    cluster is essentially a generating function of compatible pairs.  The
    motivation for the definition is to guarantee certain desired positivity
    properties in a minimal, ``greedy'' way.

    More recently, Gross-Hacking-Keel-Kontsevich introduced another basis, the
    \emph{theta basis}, based on ideas coming from mirror symmetry.  Specifically, the
    basis is defined in terms of combinatorial objects, scattering diagrams and
    broken lines, which heuristically are combinatorial avatars of various
    aspects of the SYZ fibration of the mirror cluster variety.  
    
    In rank 2, it was known that for the cluster structure associated with the
    2-Kronecker quiver these bases coincide, raising the question of extending
    this to arbitrary rank 2 cluster algebras.

    To prove this, we exploited certain uniqueness properties of the Newton
    polygons of greedy basis elements.  Ultimately, one can show that any basis
    satisfying certain strong positivity properties and whose Newton polygons
    lie within certain bounds must coincide with the greedy basis.  In turn, we
    found we could show that the Newton polygons of theta basis elements
    satisfied these bounds, which in combination with positivity results of
    Gross-Hacking-Keel-Kontsevich lead to the result. 
    
    Our construction, even though succeeds in establishing the desired
    equivalence, it is not entirely satisfactory. Ideally, given an element $x$
    in these basis, one would like to have an explicit bijection relating the 
    broken lines, appearing when considering $x$ as an element of a theta basis,
    to the compatible pairs that build $x$ as an element of a greedy basis.

    Such a bijection would yield new insights into the combinatorial structures
    of the theta basis. In particular, from partial results that we already
    have, in some cases we can give direct combinatorial interpretations in
    terms of compatible pairs to previously mysterious coefficients appearing in
    rank 2 scattering diagrams.
    
    This SQuaRE proposal is about constructing this explicit bijection.
    \sswarning{I do not like this conclusion but I can't come up with anything
      nicer right now.}

\end{document}
