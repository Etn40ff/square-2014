\documentclass{amsart}
\usepackage[margin=1in]{geometry}

\title[\small{Scattering Diagrams and Greedy Bases for Rank 2 Cluster Algebras}]
  {AIM SQuaRE Proposal\\ 
  \small{Scattering Diagrams and Greedy Bases for Rank 2 Cluster Algebras}}

\begin{document}
  \maketitle
  
  \section*{Overview}
    The goal of this AIM SQuaRE Proposal is to complete a project started at the
    AMS Mathematics Research Community on Cluster Algebras in June 2014.
    During the MRC we proved the equivalence of two a priori distinct bases of
    rank 2 cluster algebras: the \emph{greedy basis} of Lee-Li-Zelevinsky and
    the \emph{theta basis} of Gross-Hacking-Keel-Kontsevich.
    The proof suggests a number of nontrivial combinatorial refinements we will
    pursue during the SQuaRE.
    In particular, our goal is to derive novel enumerative interpretations for
    quantities appearing in the construction of the theta basis using
    combinatorial objects appearing in the construction of the greedy basis.

  \subsection*{Participants}
    The participants to the meeting will be 
    \begin{itemize}
      \item Maria Angelica Cueto (NSF Postdoctoral Fellow, Columbia University)
      \item Gregory Muller (Assistant Professor, University of Michigan)
      \item Gregg Musiker (Assistant Professor, University of Minnesota)
      \item Dylan Rupel (Postdoctoral Teaching Associate, Northeastern
          University)
      \item Salvatore Stella (Postdoc, North Carolina State University;
          INdAM-Cofund research fellow)
      \item Harold Williams (RTG Postdoctoral Instructor, University of 
          Texas at Austin) 
    \end{itemize}
    In addition to the above the original MRC group included
    \begin{itemize}
      \item Man Wai Cheung (Ph.D. Candidate, UCSD Department of Mathematics;
          Visiting student, University of Cambridge)
      \item Mark Gross (Professor, University of Cambridge)
    \end{itemize}
    Even if they will not be present at this SQuaRE meeting they remain integral
    members of this study group. 
    Indeed, should this project be funded, we plan on keeping them constantly
    updated on our progress with written status reports and, if possible, via
    short web conferences.
  
  \subsection*{Timeline} 
    Should this proposal be funded we would like to hold our meeting in the
    second half of June 2015.

  \section*{Project description}

    A central issue in the theory of cluster algebras is to define a natural
    ``canonical'' basis of a cluster algebra.
    Indeed, the original inspiration for the discovery of cluster algebras was
    the recognition of certain combinatorial structures underlying Lusztig's
    dual canonical basis in the coordinate ring of a unipotent group.
    One of the goals of the subject is to understand this particular basis on a
    more abstract level and generalize it to other cluster algebras.
    There are by now many different approaches to this problem, leading to a
    number of different bases in any given cluster algebra.
    During the MRC we proved that two of these bases for rank 2 cluster algebras
    coincide in a nontrivial way.

    One of these, the \emph{greedy basis} of Lee-Li-Zelevinsky, is described in
    terms of combinatorial objects called compatible pairs.
    Each greedy basis element is associated with a Dyck path in some rectangular
    grid, and a compatible pair is a configuration of subpaths satisfying
    certain conditions.
    The expansion of the basis element as a Laurent polynomial in a collection
    of initial cluster variables is a generating function of compatible pairs.
    The motivation for the definition is to guarantee certain desired positivity
    properties in a minimal, ``greedy'' way.

    More recently, Gross-Hacking-Keel-Kontsevich introduced another basis, the
    \emph{theta basis}, based on ideas from mirror symmetry.
    This basis is defined in terms of combinatorial objects, scattering diagrams
    and broken lines, which heuristically are avatars of various aspects of the
    SYZ fibration of the mirror cluster variety.
    
    To prove these bases coincide, we exploit certain uniqueness properties of
    the Newton polygons of greedy basis elements.
    Ultimately, one can show that any basis satisfying certain strong positivity
    properties and whose Newton polygons lie within certain bounds must coincide
    with the greedy basis.
    In turn, we found we could show that the Newton polygons of theta basis
    elements satisfied these bounds, which in combination with positivity
    results of Gross-Hacking-Keel-Kontsevich lead to the result.
    
    This equivalence immediately suggests a stronger combinatorial result.
    Ideally, given an element $x$ of these bases, one would like to have an
    explicit bijection relating the broken lines that appear when considering
    $x$ as an element of a theta basis to the compatible pairs that build $x$ as
    an element of a greedy basis.
    Such a bijection would yield new insights into the combinatorial structures
    of the theta basis.
    In particular, in writing an element of the theta basis, broken lines are
    counted with previously mysterious coefficients, and this result would
    provide a direct combinatorial interpretation of these coefficients.
    
    From our work at the MRC, we have several cases where we can construct the
    desired bijection between broken lines and compatible pairs.
    In the SQuaRE we have proposed, we would continue this work and deepen our
    understanding of these two important constructions.
    We also plan to pursue generalizations of these results for higher rank
    cluster algebras and quantum cluster algebras.

  \begin{thebibliography}{XXX}
    \bibitem{fomin-zelevinsky1}
      S.~Fomin and A.~Zelevinsky, Cluster algebras I: Foundations,
      \textsl{J. Amer. Math. Soc.} \textbf{15} (2002), no. 2, pp.~497--529.

    \bibitem{canonical}
      M.~Gross, P.~Hacking, S.~Kell and M.~Kontsevich: Canonical bases for
      cluster algebras, in preparation (2014).

    \bibitem{GS11}
      M. Gross and B. Siebert: From affine geoemtry to complex geometry,
      \textsl{Ann. Math.} \textbf{174} (2011), 1301--1428.

    \bibitem{lee-li-zelevinsky}
      K.~Lee, L.~Li, and A.~Zelevinsky, Greedy elements in rank 2 cluster
      algebras, \textsl{Selecta Math.} \textbf{20} (2014), 57--82.

    \bibitem{lee-li-zelevinsky2}
      K. Lee, L. Li and A. Zelevinsky, Positivity and tameness in rank 2 cluster
      algebras, \textsl{J. Alg. Comb.} 2014, doi: 10.1007/s10801-014-0509-6.

    \bibitem{sherman-zelevinsky}
      P.~Sherman and A.~Zelevinsky, Positivity and Canonical Bases in Rank 2
      Cluster Algebras of Finite and Affine Types, \textsl{Mosc. Math. J.}
      \textbf{4} (2004), no. 4, 947--974.

  \end{thebibliography}

\end{document}
